% Options for packages loaded elsewhere
\PassOptionsToPackage{unicode}{hyperref}
\PassOptionsToPackage{hyphens}{url}
%
\documentclass[
]{article}
\title{Snack Bar Analysis}
\author{Will Sheehan}
\date{24/12/2021}

\usepackage{amsmath,amssymb}
\usepackage{lmodern}
\usepackage{iftex}
\ifPDFTeX
  \usepackage[T1]{fontenc}
  \usepackage[utf8]{inputenc}
  \usepackage{textcomp} % provide euro and other symbols
\else % if luatex or xetex
  \usepackage{unicode-math}
  \defaultfontfeatures{Scale=MatchLowercase}
  \defaultfontfeatures[\rmfamily]{Ligatures=TeX,Scale=1}
\fi
% Use upquote if available, for straight quotes in verbatim environments
\IfFileExists{upquote.sty}{\usepackage{upquote}}{}
\IfFileExists{microtype.sty}{% use microtype if available
  \usepackage[]{microtype}
  \UseMicrotypeSet[protrusion]{basicmath} % disable protrusion for tt fonts
}{}
\makeatletter
\@ifundefined{KOMAClassName}{% if non-KOMA class
  \IfFileExists{parskip.sty}{%
    \usepackage{parskip}
  }{% else
    \setlength{\parindent}{0pt}
    \setlength{\parskip}{6pt plus 2pt minus 1pt}}
}{% if KOMA class
  \KOMAoptions{parskip=half}}
\makeatother
\usepackage{xcolor}
\IfFileExists{xurl.sty}{\usepackage{xurl}}{} % add URL line breaks if available
\IfFileExists{bookmark.sty}{\usepackage{bookmark}}{\usepackage{hyperref}}
\hypersetup{
  pdftitle={Snack Bar Analysis},
  pdfauthor={Will Sheehan},
  hidelinks,
  pdfcreator={LaTeX via pandoc}}
\urlstyle{same} % disable monospaced font for URLs
\usepackage[margin=1in]{geometry}
\usepackage{color}
\usepackage{fancyvrb}
\newcommand{\VerbBar}{|}
\newcommand{\VERB}{\Verb[commandchars=\\\{\}]}
\DefineVerbatimEnvironment{Highlighting}{Verbatim}{commandchars=\\\{\}}
% Add ',fontsize=\small' for more characters per line
\usepackage{framed}
\definecolor{shadecolor}{RGB}{248,248,248}
\newenvironment{Shaded}{\begin{snugshade}}{\end{snugshade}}
\newcommand{\AlertTok}[1]{\textcolor[rgb]{0.94,0.16,0.16}{#1}}
\newcommand{\AnnotationTok}[1]{\textcolor[rgb]{0.56,0.35,0.01}{\textbf{\textit{#1}}}}
\newcommand{\AttributeTok}[1]{\textcolor[rgb]{0.77,0.63,0.00}{#1}}
\newcommand{\BaseNTok}[1]{\textcolor[rgb]{0.00,0.00,0.81}{#1}}
\newcommand{\BuiltInTok}[1]{#1}
\newcommand{\CharTok}[1]{\textcolor[rgb]{0.31,0.60,0.02}{#1}}
\newcommand{\CommentTok}[1]{\textcolor[rgb]{0.56,0.35,0.01}{\textit{#1}}}
\newcommand{\CommentVarTok}[1]{\textcolor[rgb]{0.56,0.35,0.01}{\textbf{\textit{#1}}}}
\newcommand{\ConstantTok}[1]{\textcolor[rgb]{0.00,0.00,0.00}{#1}}
\newcommand{\ControlFlowTok}[1]{\textcolor[rgb]{0.13,0.29,0.53}{\textbf{#1}}}
\newcommand{\DataTypeTok}[1]{\textcolor[rgb]{0.13,0.29,0.53}{#1}}
\newcommand{\DecValTok}[1]{\textcolor[rgb]{0.00,0.00,0.81}{#1}}
\newcommand{\DocumentationTok}[1]{\textcolor[rgb]{0.56,0.35,0.01}{\textbf{\textit{#1}}}}
\newcommand{\ErrorTok}[1]{\textcolor[rgb]{0.64,0.00,0.00}{\textbf{#1}}}
\newcommand{\ExtensionTok}[1]{#1}
\newcommand{\FloatTok}[1]{\textcolor[rgb]{0.00,0.00,0.81}{#1}}
\newcommand{\FunctionTok}[1]{\textcolor[rgb]{0.00,0.00,0.00}{#1}}
\newcommand{\ImportTok}[1]{#1}
\newcommand{\InformationTok}[1]{\textcolor[rgb]{0.56,0.35,0.01}{\textbf{\textit{#1}}}}
\newcommand{\KeywordTok}[1]{\textcolor[rgb]{0.13,0.29,0.53}{\textbf{#1}}}
\newcommand{\NormalTok}[1]{#1}
\newcommand{\OperatorTok}[1]{\textcolor[rgb]{0.81,0.36,0.00}{\textbf{#1}}}
\newcommand{\OtherTok}[1]{\textcolor[rgb]{0.56,0.35,0.01}{#1}}
\newcommand{\PreprocessorTok}[1]{\textcolor[rgb]{0.56,0.35,0.01}{\textit{#1}}}
\newcommand{\RegionMarkerTok}[1]{#1}
\newcommand{\SpecialCharTok}[1]{\textcolor[rgb]{0.00,0.00,0.00}{#1}}
\newcommand{\SpecialStringTok}[1]{\textcolor[rgb]{0.31,0.60,0.02}{#1}}
\newcommand{\StringTok}[1]{\textcolor[rgb]{0.31,0.60,0.02}{#1}}
\newcommand{\VariableTok}[1]{\textcolor[rgb]{0.00,0.00,0.00}{#1}}
\newcommand{\VerbatimStringTok}[1]{\textcolor[rgb]{0.31,0.60,0.02}{#1}}
\newcommand{\WarningTok}[1]{\textcolor[rgb]{0.56,0.35,0.01}{\textbf{\textit{#1}}}}
\usepackage{graphicx}
\makeatletter
\def\maxwidth{\ifdim\Gin@nat@width>\linewidth\linewidth\else\Gin@nat@width\fi}
\def\maxheight{\ifdim\Gin@nat@height>\textheight\textheight\else\Gin@nat@height\fi}
\makeatother
% Scale images if necessary, so that they will not overflow the page
% margins by default, and it is still possible to overwrite the defaults
% using explicit options in \includegraphics[width, height, ...]{}
\setkeys{Gin}{width=\maxwidth,height=\maxheight,keepaspectratio}
% Set default figure placement to htbp
\makeatletter
\def\fps@figure{htbp}
\makeatother
\setlength{\emergencystretch}{3em} % prevent overfull lines
\providecommand{\tightlist}{%
  \setlength{\itemsep}{0pt}\setlength{\parskip}{0pt}}
\setcounter{secnumdepth}{-\maxdimen} % remove section numbering
\ifLuaTeX
  \usepackage{selnolig}  % disable illegal ligatures
\fi

\begin{document}
\maketitle

\begin{Shaded}
\begin{Highlighting}[]
\FunctionTok{library}\NormalTok{(readxl)}
\FunctionTok{library}\NormalTok{(ggplot2)}
\FunctionTok{library}\NormalTok{(tidyverse)}
\FunctionTok{library}\NormalTok{(janitor) }\CommentTok{\#for cleaning col names}
\NormalTok{SB\_data }\OtherTok{\textless{}{-}} \FunctionTok{read\_excel}\NormalTok{(}\StringTok{"\textasciitilde{}/Downloads/SB data.xlsx"}\NormalTok{)}
\FunctionTok{summary}\NormalTok{(SB\_data)}
\end{Highlighting}
\end{Shaded}

\begin{verbatim}
##    Order ID         Financial Status   Purchase Time                
##  Length:624         Length:624         Min.   :2021-07-30 12:16:47  
##  Class :character   Class :character   1st Qu.:2021-08-03 20:42:26  
##  Mode  :character   Mode  :character   Median :2021-08-10 20:57:12  
##                                        Mean   :2021-08-17 16:17:53  
##                                        3rd Qu.:2021-08-19 20:31:27  
##                                        Max.   :2021-09-25 16:03:03  
##                                        NA's   :2                    
##    Currency            Subtotal          Taxes            Total       
##  Length:624         Min.   : 0.220   Min.   :0.0300   Min.   : 0.250  
##  Class :character   1st Qu.: 2.210   1st Qu.:0.2900   1st Qu.: 2.500  
##  Mode  :character   Median : 3.100   Median :0.4000   Median : 3.500  
##                     Mean   : 3.632   Mean   :0.4718   Mean   : 4.104  
##                     3rd Qu.: 4.870   3rd Qu.:0.6300   3rd Qu.: 5.500  
##                     Max.   :15.920   Max.   :2.0700   Max.   :17.990  
##                                                                       
##     Quantity       Product              Price       Payment Method    
##  Min.   :1.000   Length:624         Min.   :0.220   Length:624        
##  1st Qu.:1.000   Class :character   1st Qu.:1.770   Class :character  
##  Median :1.000   Mode  :character   Median :2.210   Mode  :character  
##  Mean   :1.194                      Mean   :2.361                     
##  3rd Qu.:1.000                      3rd Qu.:3.100                     
##  Max.   :5.000                      Max.   :3.310                     
## 
\end{verbatim}

\begin{Shaded}
\begin{Highlighting}[]
\FunctionTok{ggplot}\NormalTok{(SB\_data, }\FunctionTok{aes}\NormalTok{(Quantity, Product))}\SpecialCharTok{+}
  \FunctionTok{geom\_point}\NormalTok{()}
\end{Highlighting}
\end{Shaded}

\includegraphics{Snack-Bar-Analysis_files/figure-latex/unnamed-chunk-2-1.pdf}

\hypertarget{this-chart-is-useful-when-paired-with-the-number-of-orders-where-each-item-was-purchased-chart}{%
\subsubsection{This chart is useful when paired with the ``Number of
Orders Where Each Item was Purchased''
chart}\label{this-chart-is-useful-when-paired-with-the-number-of-orders-where-each-item-was-purchased-chart}}

\begin{itemize}
\item
  While the previous chart displays the amount of orders containing any
  given item, this chart displays the quantities each item is most
  frequently purchased in (per single order). By pairing the two, we can
  see the need to have a very strong supply of popcorn, pop, freezies,
  sour keys and slushies.
\item
  Popcorn, freezies, pop and chocolate/candy are often bought in higher
  numbers This information should be factored into the restocking
  process
\end{itemize}

\begin{Shaded}
\begin{Highlighting}[]
\NormalTok{p1 }\OtherTok{\textless{}{-}} \FunctionTok{ggplot}\NormalTok{(SB\_data, }\FunctionTok{aes}\NormalTok{(purchase\_time, product))}\SpecialCharTok{+}
  \FunctionTok{geom\_point}\NormalTok{() }\SpecialCharTok{+} \FunctionTok{labs}\NormalTok{(}\AttributeTok{x =} \StringTok{"Date"}\NormalTok{, }\AttributeTok{y =} \StringTok{"Product Purchased"}\NormalTok{)}
\NormalTok{p1}
\end{Highlighting}
\end{Shaded}

\includegraphics{Snack-Bar-Analysis_files/figure-latex/unnamed-chunk-4-1.pdf}

\hypertarget{this-chart-displays-the-items-purchased-over-time}{%
\subsubsection{This chart displays the items purchased over
time}\label{this-chart-displays-the-items-purchased-over-time}}

\begin{itemize}
\tightlist
\item
  The month of August was very busy, with most items being ordered
  frequently
\item
  Hot dogs and slushies were not popular in August. This could be due to
  a lack of stock or a failure to have the item prepared
\item
  It is possible that this data would simply align with the hours of
  operation, thus rendering this chart obsolete
\end{itemize}

\hypertarget{condiderations}{%
\subsubsection{Condiderations:}\label{condiderations}}

\begin{itemize}
\tightlist
\item
  This data will be heavily influenced by when the SB is open/ when
  events are happening
\item
  This data is pulled from a range of time that was drastically skewed
  due to the pandemic, and therefore not a completely accurate
  representation of overall SB performance
\end{itemize}

\end{document}
